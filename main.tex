\documentclass[journal,article,submit,moreauthors,pdftex,10pt,a4paper]{mdpi}
\usepackage[utf8]{inputenc}
\usepackage[T1]{fontenc}
\usepackage{amsmath}
\usepackage{amssymb}
\usepackage{graphicx}
\usepackage{verbatim}
\usepackage{float}

\usepackage[portuguese]{babel}

\usepackage[nomain, acronym, symbols]{glossaries}

\loadglsentries{acronyms}

\definecolor{cadmiumgreen}{rgb}{0.0, 0.42, 0.24}
\definecolor{burntorange}{rgb}{0.8, 0.33, 0.0}

\newcommand{\ct}[1]{{\color{black}#1}}
\newcommand{\odo}[1]{{\color{black}#1}}
\newcommand{\gus}[1]{{\color{black}#1}}
\newcommand{\cris}[1]{{\color{black}#1}}

\firstpage{1}
\makeatletter
\setcounter{page}{\@firstpage}

\makeatother

\Title{Proposta e avaliação de transformadas 4D direcionais para compressão de light fields}

\begin{document}

% Light fields são uma tecnologia de imagem capaz de representar de maneira realista as informações luminosas de uma
% cena. Diferente de fotografias 2D tradicionais, que representam apenas a intensidade dos raios de luz incidentes em um
% plano, light fields estão definidos em 4D, pois também capturam o ângulo de incidência destes raios de luz, permitindo
% uma representação precisa de efeitos de paralax, refração, reflexão, volumétricos, dentre outros \cite{system_lf_vr}.
% Estas propriedades permitem que light fields sejam utilizados para visualização imersiva em telas especiais (semelhante
% ao cinema 3D, porém sem a necessidade de óculos), geração de malhas tridimensionais, inferência de mapas de
% profundidade e edições de foco mesmo após a captura da imagem \cite{lightfields_survey, ebrahimi2016jpeg}.

% JPEG Pleno é um comitê de projeto, iniciado em 2015, que faz parte do grupo de trabalho JPEG, sob a \ac{iso}/\ac{iec}
% \cite{ctc_jpeg_pleno}. Este comitê se propõe a padronizar formas de codificação e representação de imagens plenópticas.
% Dentre elas estão: Hologramas, Point Clouds e Light Fields. Um light field, de acordo com a definição adotada pelo JPEG
% Pleno, é representado como um conjunto de vistas (imagens 2D, semelhantes a fotografias tradicionais) dispostas em duas
% dimensões, que representam as duas componentes dos ângulos .

% As amostras de um ponto dentro deste modelo são convencionalmente denotadas por uma quádrupla $(t, s, v, u)$, onde $(v,
%     u)$ são respectivamente as coordenadas verticais e horizontais de cada vista, e $(t, s)$ são as componentes verticais e
% horizontais dos ângulos que esta amostras representam. Devido à alta densidade de dados necessária para representar um
% light field, dois modos de compressão com perdas foram adotados no padrão \cite{jpeg_pleno_p2, schelkens2019jpeg}. São
% eles:

% \begin{enumerate}

%     \item \ac{4dpm}, um modelo preditivo, semelhante ao modo preditivo utilizado em em codificação de vídeos, onde certas vistas
%           do light field são utilizadas como preditoras para as demais e os resíduos são posteriormente codificados através de um
%           codificador de imagens tradicionais.

%     \item \ac{4dtm}, um modelo baseado na \ac{4ddct}, em que os dados transformados são representados através de uma árvore de
%           bitplanes.

% \end{enumerate}

% A utilização de dois métodos de compressão foi adotada pois cada modo se adéqua melhor a um tipo de dado. Em light
% fields com vistas bastante semelhantes, como aqueles capturados por câmeras plenópticas, costumam ser mais
% eficientemente comprimidos através do modo \ac{4dtm}. Já aqueles em que as vistas estão mais distantes, e portanto são
% mais diferentes entre si, como os light fields capturados por matrizes de câmeras, são melhor representados através do
% modo \ac{4dpm}.

% Recentemente um novo modo de compressão baseado no \ac{4dtm} foi proposto \cite{slanted_4d_dct}. Este modelo, batizado
% de \ac{s4dtm}, explora o fato de que em um plano epipolar através das posições $(v_0, t_0)$, bordas diagonais costumam
% aparecer nos dados, o que prejudica a capacidade de concentração de energia da \ac{4ddct}. Para contornar este problema
% é feita uma transformada \textit{slant} em cada bloco. Esta transformada distorce o bloco de modo que a disposição de
% seus dados fique orientada na vertical, aumentando assim a eficiência de compactação da \ac{4ddct} que será aplicada em
% seguida. Testes realizados demonstram que o \ac{s4dtm} superou os modos \ac{4dtm} e \ac{4dpm} com ganhos em BD-Rate de
% 31.03\% and 28.30\%, respectivamente, implicando na possibilidade de que um único modo de codificação seja adotado no
% padrão.

% Para realizar a \ac{4ddct} é necessário que os dados de entrada estejam organizados como hyper-paralelepípedos, porém o
% resultado da transformada \textit{slant} são hyper-trapézios. Este problema exige que um passo de preenchimento de
% áreas vazias (\textit{padding}) seja aplicado gerando hyper-paralelepípedos maiores do que os originais, porém com
% características mais adequadas para a \ac{4ddct}, de modo que o ganho de eficiência de codificação compensa a inserção
% extra de dados.

% Apesar dos ganhos significativos em eficiência de codificação, a necessidade de inserir dados para o \textit{padding}
% sugere que este modelo ainda possui margem para melhorias. No caso bidimensional, o problema da eficiência da
% \ac{2ddct} em dados dispostos na diagonal já foi tema de diversos trabalhos. Em codificadores de vídeo como o \ac{hevc}
% e \ac{vvc} o problema é mitigado utilizando preditores diagonais no modo de predição intra quadro \cite{overview_hevc,
% overview_vvc, non_separable_transform}. Outros trabalhos propuseram diferentes modificações na própria \ac{dct},
% utilizando transformadas similares ou modificando a disposição da imagem e forma de percorrer as amostras
% \cite{directional_dct, diagonal_dct, sparse_ddct, diagonally_oriented_dct, fracastoro2016_steerable_dct,
% peloso2020_steerable_dct, lima2021_steerable_dct_3d}.

% Este projeto de mestrado tem como objetivo propor novas transformadas direcionais, bem como estender para 4 dimensões
% transformadas direcionais em 2 ou 3 dimensões já existentes, a fim de melhorar a eficiência de codificação em light
% fields. As transformadas serão inicialmente avaliadas isoladamente em termos de \textit{coding gain}, a fim de avaliar
% sua performance teórica. Em seguida, as propostas mais eficientes serão implementadas dentro do software de referência
% \ac{jplm} \cite{jplm, jpeg_pleno_p4}, onde outras características poderão ser avaliadas.

% Dentro do \ac{jplm} será possível avaliar erros introduzidos pela quantização em termos de \ac{mse}, \ac{psnr} e
% \ac{ssim}. Também serão avaliados os ganhos reais de codificação, em termos de \ac{bpp}, já considerando os custos de
% sinalização necessários para a transformada direcional. Através destas duas medidas cada light field disponível nas
% \ac{ctc}~\cite{ctc_jpeg_pleno}, será codificado utilizando cada uma das transformadas para cada taxa alvo disponível.
% As curvas de eficiência de codificação obtida com estes dados serão feitas e o BD-Rate será calculado para que os
% ganhos possam ser efetivamente comparados.

\newpage

\bibliography{references}

\end{document}
