\documentclass[journal,article,submit,moreauthors,pdftex,10pt,a4paper]{mdpi}
\usepackage[utf8]{inputenc}
\usepackage[T1]{fontenc}
\usepackage{amsmath}
\usepackage{amssymb}
\usepackage{graphicx}
\usepackage{verbatim}
\usepackage{float}

\usepackage[portuguese]{babel}

\usepackage[nomain, acronym, symbols]{glossaries}

\loadglsentries{acronyms}

\definecolor{cadmiumgreen}{rgb}{0.0, 0.42, 0.24}
\definecolor{burntorange}{rgb}{0.8, 0.33, 0.0}

\newcommand{\ct}[1]{{\color{black}#1}}
\newcommand{\odo}[1]{{\color{black}#1}}
\newcommand{\gus}[1]{{\color{black}#1}}
\newcommand{\cris}[1]{{\color{black}#1}}

\firstpage{1}
\makeatletter
\setcounter{page}{\@firstpage}

\makeatother

\Title{Proposta de um Escalonador \textit{Multicore} Adaptativo baseado em uma ANN para RTOS}

\begin{document}

\section{Introdução e Motivação}

\ac{rtos} são sistemas desenvolvidos para o processamento de tarefas com restrições temporais, onde deve-se assegurar
que os prazos de conclusão das tarefas serão cumpridos. É possível classificar esses sistemas como críticos ou não
críticos conforme as suas aplicações e requisitos. \ac{rtos} críticos são utilizados em contextos onde a perda de prazo
na execução de tarefas pode ter consequências graves, o que inclui sistemas industriais, veículos autônomos, sistemas
de navegação de aeronaves e outros. Já em \ac{rtos} não-críticos, a perda de prazos pode ser tolerada, sendo que seus
casos de uso envolvem sistemas de multimídia e processamento de áudio \cite[13--22]{kopetz2022real}.

As tarefas a serem executadas podem ter tempos de execução, prazos e requisitos de recursos computacionais variados,
sendo assim, são utilizados escalonadores que definem prioridades para cada tarefa de modo a ordená-las de uma forma em
que todas as restrições temporais sejam satisfeitas. Os escalonadores que a definem as prioridades antes da execução do
sistema são denominados estáticos, enquanto os que definem as prioridades durante o tempo de execução são considerados
dinâmicos \cite[247--248]{kopetz2022real}.

Atualmente, diversas arquiteturas de processadores são \textit{multicore}, permitindo o paralelismo na execução de
tarefas. A complexidade dessas arquiteturas introduz novos desafios na implementação de escalonadores de tempo real. O
compartilhamento de recursos, como a memória, \textit{cache} ou Entrada/Saída, por exemplo, pode afetar o tempo de
execução das tarefas de forma não determinística \cite{aceituno2023optimized}. Além disso, há problemas relacionados à
otimização e à utilização eficiente dos recursos, como o balanceamento de carga entre núcleos ou a otimização no
consumo de energia do sistema \cite{jadon2024comprehensive, baital2024energy}.

% Falar sobre IA aqui
% Propor

% ---

A proposta deste trabalho é o desenvolvimento de um algoritmo adaptativo que utiliza uma \ac{ann} para escalonar as
tarefas eficientemente com base nas métricas de desempenho do \ac{rtos}.

% Nem sempre o escalonamento é bom para a situação ...
% Pode ocorrer perdas de deadlines, o que é bem ruim em sistemas críticos ...
% Alguns algoritmos de escalonamento podem apresentar um desempenho melhor para situações específicas ...

\section{Objetivos}

\begin{itemize}
    \item A;
    \item B;
    \item C;
    \item D;
    \item E.
\end{itemize}

\section{Estado da Arte}

\section{Contribuições}

\section{Metodologia}

\newpage

\bibliography{references}

\end{document}
