\documentclass[journal,article,submit,moreauthors,pdftex,10pt,a4paper]{mdpi}
\usepackage[utf8]{inputenc}
\usepackage[T1]{fontenc}
\usepackage{amsmath}
\usepackage{amssymb}
\usepackage{graphicx}
\usepackage{verbatim}
\usepackage{float}

\usepackage[portuguese]{babel}

\usepackage[nomain, acronym, symbols]{glossaries}

\loadglsentries{acronyms}

\definecolor{cadmiumgreen}{rgb}{0.0, 0.42, 0.24}
\definecolor{burntorange}{rgb}{0.8, 0.33, 0.0}

\newcommand{\ct}[1]{{\color{black}#1}}
\newcommand{\odo}[1]{{\color{black}#1}}
\newcommand{\gus}[1]{{\color{black}#1}}
\newcommand{\cris}[1]{{\color{black}#1}}

\firstpage{1}
\makeatletter
\setcounter{page}{\@firstpage}

\makeatother

\Title{Proposta de um Escalonador Adaptativo baseado em Aprendizado de Máquina para Sistemas Operacionais de Tempo Real
\textit{Multi-core}}

\begin{document}

\section{Introdução e Motivação}

\gls{rtos} são sistemas desenvolvidos para o processamento de tarefas com restrições temporais, onde deve-se assegurar
que os prazos de conclusão das tarefas sejam cumpridos. É possível classificar esses sistemas como críticos ou não
críticos conforme as suas aplicações e requisitos. \gls{rtos} críticos são utilizados em contextos onde a perda de
prazo na execução de tarefas pode ter consequências graves, o que inclui sistemas industriais, veículos autônomos,
sistemas de navegação de aeronaves e outros. Já em \gls{rtos} não-críticos, a perda de prazos pode ser tolerada, sendo
que os seus casos de uso envolvem sistemas de multimídia e processamento de áudio \cite[13--22]{kopetz2022real}.

As tarefas a serem executadas podem ter tempos de execução, prazos e requisitos de recursos computacionais variados,
sendo assim, são utilizados escalonadores que definem prioridades para cada tarefa de modo a ordená-las de uma forma em
que todas as restrições temporais sejam satisfeitas. Os escalonadores que a definem as prioridades antes da execução do
sistema são denominados estáticos, enquanto os que definem as prioridades durante o tempo de execução são considerados
dinâmicos \cite[247--248]{kopetz2022real}.

Atualmente, diversas arquiteturas de processadores são \textit{multi-core}, permitindo o paralelismo na execução de
tarefas. A complexidade dessas arquiteturas introduz novos desafios na implementação de escalonadores de tempo real. O
compartilhamento de recursos, como a memória, \textit{cache} ou Entrada/Saída, por exemplo, pode afetar o tempo de
execução das tarefas de forma não determinística \cite{aceituno2023optimized}. Além disso, há problemas relacionados à
otimização e utilização eficiente dos recursos, como o balanceamento de carga entre núcleos ou a otimização no consumo
de energia do sistema \cite{jadon2024comprehensive, baital2024energy}.

Considerando este cenário, diferentes técnicas de escalonamento foram propostas nos últimos anos. Algumas técnicas
utilizam heurísticas, como algoritmos genéticos, para otimizar o desempenho do sistema. Há também a aplicação de
\gls{ml}, principalmente com técnicas de \gls{rl} \cite{hassan2025optimizing, liang2024adaptive}.

Propõe-se então o desenvolvimento de um escalonador dinâmico adaptativo para \gls{rtos} \textit{multi-core} que utilize
técnicas de \gls{ml} para a otimização do desempenho. A priorização das tarefas seria feita com base nas métricas do
sistema, permitindo que elas sejam executadas de uma forma que considere as variações de carga, perdas de prazo, e
outros fatores relevantes que afetam o tempo de execução.

\section{Objetivos}

Os objetivos principais incluem:

\begin{itemize}
    \item Realizar uma revisão sistemática sobre a aplicação de \gls{ml} na área de algoritmos de escalonamento para
          \gls{rtos};
    \item Desenvolver a arquitetura do escalonador e investigar técnicas de treinamento no contexto de \gls{ml} para
          algoritmos de escalonamento;
    \item Realizar a validação do escalonador com diferentes cenários de teste;
    \item Analisar os resultados obtidos, comparando o escalonador com outros algoritmos de escalonamento relevantes.
\end{itemize}

\section{Estado da Arte}

No cenário de escalonamento para \gls{rtos} \textit{multi-core}, diversas propostas para problemas pertinentes foram
feitas, muitas delas focando em sistemas críticos em ambientes embarcados \cite{ismael2021scheduling}.

Um dos desafios introduzidos pelas arquiteturas \textit{multi-core} é a distribuição eficiente da carga de trabalho
entre os núcleos. Muitas das soluções propostas ainda são limitadas por problemas como a heterogeneidade das
arquiteturas, escalabilidade, dinamicidade das cargas de trabalho e \textit{overheads} causados pela complexidade dos
escalonadores \cite{jadon2024comprehensive}. A aplicação de \gls{ml} no contexto geral escalonamento
\textit{multi-core} ainda é limitada, mas há propostas que aplicam \gls{rl} e \gls{drl} \cite{liang2024adaptive,
tan2024deep}.

Outro tópico relevante é a segurança desses sistemas, principalmente em ambientes distribuídos. Neste contexto, os
aspectos de segurança, como a confidencialidade, autenticidade e a integridade dos dados devem ser considerados em
conjunto com as restrições temporais do sistema. A aplicação de \gls{ml} nessa linha ainda é baixa
\cite{singh2023systematic}.

Com a aplicação de sistemas de tempo real em ambientes embarcados com fontes limitadas de energia, a eficiência
energética se tornou um tópico relevante na área. Vários trabalhos propostos nessa linha utilizam \gls{ml}, aplicando
\gls{drl}, redes de \gls{lstm}, regressão linear e \glspl{cnn} \cite{peng2024energy, allaqband2024efficient,
khan2025enhanced}.

Para sistema de tempo real em geral, a aplicação de \gls{ml} em algoritmos de escalonamento ainda é recente, mas ela se
demonstra promissora, indicando que há espaço para avanços no estado da arte.

\section{Contribuições}

Com a execução deste trabalho, espera-se contribuir com:

\begin{itemize}
    \item A discussão e resultados de uma revisão sistemática sobre a aplicação de \gls{ml} na área de algoritmos de
          escalonamento para \gls{rtos} \textit{multi-core};
    \item Um algoritmo adaptativo baseado em \gls{ml} para o escalonamento em \gls{rtos} \textit{multi-core};
    \item Resultados de testes com o algoritmo desenvolvido e análises comparativas.
\end{itemize}

\section{Metodologia}

O projeto se trata de uma pesquisa aplicada e será iniciado com uma pesquisa exploratória, visando o desenvolvimento da
fundamentação teórica e o estabelecimento do estado da arte. Posteriormente, a arquitetura do escalonador será definida
e o seu desenvolvimento será planejado. Os passos mais relevantes no desenvolvimento envolvem a obtenção dos dados de
treinamento, o treinamento em si e a validação. Por fim, será feita a análise comparativa dos resultados.

\newpage

\bibliography{references}

\end{document}
