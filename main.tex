\documentclass[journal,article,submit,moreauthors,pdftex,10pt,a4paper]{mdpi}
\usepackage[utf8]{inputenc}
\usepackage[T1]{fontenc}
\usepackage{amsmath}
\usepackage{amssymb}
\usepackage{graphicx}
\usepackage{verbatim}
\usepackage{float}

\usepackage[english]{babel}

\usepackage[nomain, acronym, symbols]{glossaries}

\loadglsentries{acronyms}

\definecolor{cadmiumgreen}{rgb}{0.0, 0.42, 0.24}
\definecolor{burntorange}{rgb}{0.8, 0.33, 0.0}

\newcommand{\ct}[1]{{\color{black}#1}}
\newcommand{\odo}[1]{{\color{black}#1}}
\newcommand{\gus}[1]{{\color{black}#1}}
\newcommand{\cris}[1]{{\color{black}#1}}

\firstpage{1}
\makeatletter
\setcounter{page}{\@firstpage}

\makeatother

\Title{Machine Learning Guided Adaptive Scheduling for Multi-core Real-time Operating Systems}

\begin{document}

\section{Introduction}

% TODO: Introduction
    % TODO: Scope
    % TODO: Structure
% TODO: Methodology
% TODO: Chronogram
% TODO: Expected results

% TODO: Check the sources
% TODO: Check the correctness, structure and flow
% TODO: Check the grammar
% TODO: Change the template to IEEE single column

% Context and relevancy
% TODO: Add sources for the utilization of RTOS in these areas
Currently, several areas of technology require software that must execute in real-time. This often includes autonomous vehicles, aviation, robotics, industrial systems, and other critical systems where the software is directly responsible for system safety and control. In these domains, a failure to meet timing constraints during execution can lead to severe consequences. Thus, to ensure predictable execution and meet these timing requirements, \gls{rtos} are utilized.

% Definition of RTOS
\gls{rtos} are systems developed to handle the deterministic processing of tasks, where it is necessary to ensure that the deadlines for the tasks are met. These systems can be classified as hard or soft based on their capabilities and requirements. Hard \gls{rtos} are used in situations where the failure to meet a deadline in the execution of a task can have severe consequences, this includes the areas cited previously. Alternatively, soft \gls{rtos} are used in non-critical contexts, where deadline misses can be tolerated to a certain degree. Multimedia and audio processing are examples of contexts where a soft \gls{rtos} might be used \cite[13--22]{kopetz2022real}.

% Definition of scheduling algorithms
The deterministic execution of tasks is ensured by real-time scheduling algorithms, which are responsible for their prioritization. These algorithms must order the tasks in such a way that their timing constraints are satisfied, for that, the execution time, deadlines and other aspects of tasks can be considered. Schedulers that set the priorities before the execution of the system are classified as static, while those that set the priorities at runtime are classified as dynamic \cite[247--248]{kopetz2022real}.

\subsection{Research Problem}

% Introduction to the problems
% TODO: Check if it's ok to say the execution is faster
Most modern processors adopt a multi-core implementation, which allows tasks to be executed in parallel. This design offers many advantages, such as the higher efficiency with a faster execution speed and improved power efficiency. However, multi-core processors also introduce several challenges \cite{sethi2015multicore}.

% Problems
One of the major challenges introduced by multi-core processors is interference, which arises from task contention for shared resources such as main memory, caches, and I/O peripherals. This contention introduces non-deterministic delays into task execution, making it difficult to accurately estimate the \gls{wcet}. This interference affects the scheduler's ability to guarantee that tasks will meet their deadlines, thus compromising system reliability \cite{aceituno2023optimized, lugo2022survey}. Moreover, heterogeneous multi-core processors also introduce challenges, since they present cores with different characteristics from each other. This requires more sophisticated schedulers and is a subject of attention for load balancing mechanisms. Furthermore, the dynamism and variation of tasks' workloads is another challenge. Tasks may evolve through time, changing their resource utilization and demands, which might require adaptive schedulers \cite{mrabet2024scheduling, jadon2024comprehensive}.

% SOTA
% TODO: Verify if "classical optimization methods" is ok
Several works in the field of \gls{rtos} proposed methods to solve pertinent problems introduced by multi-core systems, most of them in the context of embedded systems \cite{ismael2021scheduling}. Some works employ optimization methods based on \gls{ilp} or \gls{pso} in order to minimize interference and optimize load balance \cite{aceituno2023optimized, liu2021multi}. However, beyound the application of heuristics and classical optimization methods, there is also the utilization of \gls{ml}.

% Machine learning
Research works that employ \gls{ml} focuses on several objectives, such as the optimization of schedulers through \gls{rl} and also in optimization for heterogeneous architectures through \gls{drl} \cite{liang2024adaptive, tan2024deep}. Power efficiency is also relevant, with several works applying \gls{ml} with techniques such as \gls{drl}, \gls{lstm} networks, linear regression and \glspl{cnn} \cite{peng2024energy, allaqband2024efficient, khan2025enhanced}. These solutions apply \gls{ml} in order to model the behavior of the system and applications, analyzing performance metrics and relevant data about tasks. With these models, it is possible to make decisions about prioritization.

% Gap
Nonetheless, the research in the area with \gls{ml} application is still limited. The dynamic context of task creation, destruction and workload variation through time is underexplored. Moreover, there is a lack of research on utilization of both hardware and software performance metrics for system state capture, specially considering multiple architectures, heterogeneous multi-core processors and highly dynamic applications.

\subsection{Research Question}

The current scenario in the state of the art leads to questions that are still unanswered:

% TODO: Improve these questions
% TODO: Check if these need to be only one question
\begin{itemize}
    \item How can a \gls{ml} model learn to capture the evolving state of the dynamic environment of task creation, evolution and destruction based on both hardware and software metrics?
    \item Will such model operate accross multiple architectures while still performing satisfactorily?
\end{itemize}

\subsection{Objectives}

% TODO: Connect this parts better
% TODO: Check if inference is ok
The general objective of this research is to develop an adaptive \gls{ml} guided scheduler for multi-core \gls{rtos}, aiming to improve the reliability of the system with dynamic applications. The model would utilize hardware and software metrics in order to make inferences about the system state, that would then be utilized for informed scheduling decitions.

\subsubsection{Specific Objectives}

\begin{itemize}
    \item Contribute with an analysis of the most relevant hardware and software performance metrics for real-time scheduling;
    \item Contribute with an analysis of the architecture's impact in the performance of the \gls{ml} model; % TODO: Maybe reword this
	\item Present an analysis of the impact of the overhead introduced by \gls{ml} models in \gls{rtos}.
\end{itemize}

\subsection{Motivation}

Considering the open problems in the field, this research has the potential to offer a new methods to address the classical challenges in real-time scheduling for multi-core systems and also answer questions related to the feasibility of the implementation of \gls{ml} models in \gls{rtos}. Furthermore, the improvement of reliability of \gls{rtos} also leads to safety improvements in several areas, such as in autonomous and industrial systems.  % TODO: Make this less capitalistic

\subsection{Scope}

\subsection{Structure}

% TODO: Add new ideas and the work of the first two semesters
\section{Methodology}

This work will be conducted as an applied research and will approach the problem with qualitative analyses. The
research will also take an exploratory approach to the subject, identifying the viability, advantages and disadvantages
of the proposed solution.

The initial step is comprised of a systematic literature review that will explore the state of the art in the
application of \gls{ml} in multi-core \gls{rtos}, with a special focus in the context of scheduling algorithms.
Moreover, the next step will be the creation of datasets of metrics collected during the execution of tasks in evolving
scenarios, including workload variation, usage of synchronizers, input/output and other aspects. After this step, the
model and scheduler will be developed with the application of concepts from the work of \citet{horstmann2019framework}.
The solution will be validated with practical experiments and data collection on \gls{epos} \cite{epos}. Lastly, the
results will be analyzed, where it will be possible to answer the research questions.

\section{Theoretical Foundation}

\subsection{Real-time Operating Systems}

\subsection{Real-time Schedulers}

\subsection{Multi-core processors}

\subsection{Machine Learning for Embedded Systems}

\newpage

\bibliography{references}

\end{document}
